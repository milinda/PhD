\documentclass{article}
\usepackage{graphicx}
\usepackage[urw-garamond]{mathdesign}
\usepackage[T1]{fontenc}
\usepackage{hyperref}

\begin{document}

\title{Weekly Report - 09/08/2014 to 09/12/2014}
\author{Milinda Pathirage}

\maketitle

\section{Distributed Stream Processing Framework Performance Study}

Didn't get that much time during last week to work on the paper. But I
worked on reading some related materials~\cite{desurvey}~\cite{dimsonhailstorm}~\cite{guo2013well} and writing the introduction
of the paper. I got the idea for the paper from the IPDPS
paper~\cite{guo2013well}.

Studying performance and other characteristics of distributed stream
processing frameworks is interesting because availability of multiple
frameworks~\cite{toshniwal2014storm}~\cite{asf:2014:samza}~\cite{neumeyer2010s4} ~\cite{zaharia2013discretized,}~\cite{akidau2013millwheel}
from different companies with each having different view
about the way to tackle this problem and differences in the internal
architecture of these frameworks. For example, Spark Streaming is
developed on top of Spark's RDD~\cite{zaharia2012resilient} data
structure and operates on batches of tuples where as Apache Storm
allows developer to process individual tuples at the lowest level and
batches using high-level programming API. I want to explore the
implications these different architectures have on different types of
stream processing topologies and different types of streaming data.


\section{HTRC}

\begin{itemize}
  \item Adding \textbf{Single Log Out} support to
    pac4j(\url{https://github.com/milinda/pac4j/tree/pac4j-1.5.1-htrc})
    open source library to use with HTRC Portal. Samitha tried this
    library for Single Sign On and found out that \textit{Single Log Out} and
    \textit{SAML2 OAuth2 Grant Type} are not available in this library. Even
    though these extra functionality is not there this is the only
    library she found which has other required functionalities and
    ease of use.  I am patching this library to add missing
    functionality so that Samitha can use it in Portal.
  \item Designed the IPython Cluster management API for HTRC IPython
    support and start implementing the API.
  \item Designed HTRC Github Project Structure. Couldn't send it to the
    tech list yet. Planning to send it to IU HTRC list first with
    detailed description. Below is the source code structure for
    Github:
    \begin{itemize}
    \item HTRC-Community: Samples, UnCamp Code, etc. comes under this.
    \item Deprecated: All the old code (currently unused) goes here
    \item HTRC-Commons: Reusable Libraries for HTRC Platform. One good example is OAuth2 Servelet Filter.
    \item HTRC-Portal
    \item HTRC-BlackLight
    \item HTRC-Bookworm
    \item HTRC-DataAPI
    \item HTRC-SolrProxy
    \item HTRC-CoreServices: Services and APIs internal to HTRC. Agent, Registry API goes here.
    \item HTRC-ToolsAndUtilities: Internal tools like log-analyzer will reside here.
    \end{itemize}
    Main idea behind above structure is having separate Github
    projects for public facing components and group internal
    components based on their functionality.
\end{itemize}

\bibliographystyle{abbrv}
\bibliography{update}

\end{document}
