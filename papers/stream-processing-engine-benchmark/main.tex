\documentclass{sig-alternate}

\usepackage[urw-garamond]{mathdesign}
\usepackage[T1]{fontenc}

% SIG Alternate copyright removal
\usepackage{etoolbox}
\makeatletter
\patchcmd{\maketitle}{\@copyrightspace}{}{}{}
\makeatother

\usepackage[usenames,dvipsnames]{color}


\begin{document}
%
% --- Author Metadata here ---
\CopyrightYear{2014} % Allows default copyright year (20XX) to be over-ridden - IF NEED BE.
%\crdata{0-12345-67-8/90/01}  % Allows default copyright data (0-89791-88-6/97/05) to be over-ridden - IF NEED BE.
% --- End of Author Metadata ---

\title{Benchmark Suite to Evaluate Performance of Distributed Stream
  Processing Platforms}

\numberofauthors{2}

\author{
% 1st. author
\alignauthor
Milinda Pathirage\\
       \affaddr{School of Informatics and Computing}\\
       \affaddr{Indiana University}\\
       \email{mpathira@indiana.edu}
% 2nd. author
\alignauthor
Beth Plale\\
       \affaddr{School of Informatics and }\\
       \affaddr{Indiana University}\\
       \email{plale@cs.indiana.edu}
}

\maketitle
\begin{abstract}
Interest on real-time query processing and processing of continuous
streams of data has increased over last couple of years due to the need
of derving actionable information as soon as possible
 to be competitive in the fast moving world. As a result of the limitations
 in batch processing technologies from previous generation, distributed
 stream processing systems like Yahoo's \textit{S4}, Twitter's \textit{Storm}, \textit{Spark
 Streaming} were introduced into the fast growing Big Data eco-system.
 Even though there are various different stream processing platforms
 and frameworks on top of them with different capabilities and
 characteristics, in-depth comparative studies of performance, scalability and
 reliability has never been done. Users of these system often face
 difficulties when choosing a system as a solution to a task at hand.
 In this paper we use some of the popular distributed stream
 processing use cases we identified by going through mailing list
 discussions, presentations and publication to evaluate three (
 \textit{Storm, Spark Streaming, Samza}) of the
 most popular distributed processing systems used widely today.
\end{abstract}

% A category with the (minimum) three required fields
%\category{Data streams}{Please fix.}{Please fix.}
%A category including the fourth, optional field follows...
%\category{D.2.8}{Please fix.}{Metrics}[complexity measures,
%performance measures, please fix.]

%\terms{Fix this.}

%\keywords{Fix this.}

\section{Introduction}
Real-time stream processing is used in various scenarios
expanding across social network monitoring, fraud detection, real-time
analytics, electronic trading, sensor networks, military applications
and internet of things applications. But there is a increasing
interest on real-time stream processing in the context of internet
applications such as social networks, electronic commerce and cloud
monitoring. Some of the big internet companies have invented their own
distributed stream processing frameworks~\cite{akidau2013millwheel}~\cite{murray2013naiad}  to be
competitive with each other or they are collaborating through open source foundations to
build next generation distributed stream processing frameworks~\cite{asf:2014:samza}~\cite{neumeyer2010s4}~\cite{toshniwal2014storm}. Apache
Storm and Apache Samza~\cite{asf:2014:samza} are two examples for collaborative efforts from
vaiours internet age companies. 

For example
Yahoo~\cite{hortonworks:2014:storm} has one of the biggest Apache Storm deployment in the
world, Twitter is using Apache Storm in their hybrid analytics
platform TSAR~\cite{twitter:2014:tsar}. 

With this push from large organizations, and contributions from some
of these organizations like Twitter~\cite{toshniwal2014storm},
Yahoo~\cite{neumeyer2010s4}, LinkedIn and as a
result of research~\cite{zaharia2013discretized}
from academic institutions like University of California, Berkeley
there are multiple popular open source distributed processing frameworks
available today. And lots of other organizations have started to utilize
 these event stream processing frameworks in their IT infrastructures.
 \textcolor{Red}{\textbf{Fix above.}}

Due to various differences in business requirements and existing
IT infrastructure in place, above mentioned event stream processing
systems were designed and implemented with different properties. They
each support different programming models, has different gurantees about
 scalability, fault tolerance, and etc. And some~\cite{asf:2014:samza} were implemented based on
 specific messaging infrastructures. New users, often faced various difficulties
 when trying to select one of these technologies to solve their own problems.
 This is mainly due to lack of comparative studies on these systems.

 \textcolor{Red}{\textbf{Talked about existing streaming benchmarks.}}

 \textcolor{Red}{\textbf{Define what we are going to do.}}

 \textcolor{Red}{\textbf{Our contributions.}}

 \textcolor{Red}{\textbf{Structure of the paper.}}

 \subsection{Event Stream Processing Tasks}


From~\cite{streamdrill:presentation}

 \textbf{Tasks by Complexity(Increasing Order)}

 \begin{itemize}
  \item Counting, Averages and Count Distinct
  \item Profiles and Histograms
  \item Trends
  \item Outliers and Fraud Detection
  \item Prediction (churn, failure)
 \end{itemize}

\textbf{Tasks by Latency(Decreasing Order)}
\begin{itemize}
 \item Reporting
 \item Visualization and Monitoring
 \item Optimizing, Personalizations
 \item Control
\end{itemize}


\begin{itemize}

  \item Why stream processing is widely used now?
  \item In which domains, stream processing is used.
  \item Why it is important to benchmark or do a emperical study of
    stream processing engines.
  \item What are our contributions
\end{itemize}

\bibliographystyle{abbrv}
\bibliography{main}
\end{document}
