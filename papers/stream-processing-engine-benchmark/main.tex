% This is "sig-alternate.tex" V2.0 May 2012
% This file should be compiled with V2.5 of "sig-alternate.cls" May 2012
%
% This example file demonstrates the use of the 'sig-alternate.cls'
% V2.5 LaTeX2e document class file. It is for those submitting
% articles to ACM Conference Proceedings WHO DO NOT WISH TO
% STRICTLY ADHERE TO THE SIGS (PUBS-BOARD-ENDORSED) STYLE.
% The 'sig-alternate.cls' file will produce a similar-looking,
% albeit, 'tighter' paper resulting in, invariably, fewer pages.
%
% ----------------------------------------------------------------------------------------------------------------
% This .tex file (and associated .cls V2.5) produces:
%       1) The Permission Statement
%       2) The Conference (location) Info information
%       3) The Copyright Line with ACM data
%       4) NO page numbers
%
% as against the acm_proc_article-sp.cls file which
% DOES NOT produce 1) thru' 3) above.
%
% Using 'sig-alternate.cls' you have control, however, from within
% the source .tex file, over both the CopyrightYear
% (defaulted to 200X) and the ACM Copyright Data
% (defaulted to X-XXXXX-XX-X/XX/XX).
% e.g.
% \CopyrightYear{2007} will cause 2007 to appear in the copyright line.
% \crdata{0-12345-67-8/90/12} will cause 0-12345-67-8/90/12 to appear in the copyright line.
%
% ---------------------------------------------------------------------------------------------------------------
% This .tex source is an example which *does* use
% the .bib file (from which the .bbl file % is produced).
% REMEMBER HOWEVER: After having produced the .bbl file,
% and prior to final submission, you *NEED* to 'insert'
% your .bbl file into your source .tex file so as to provide
% ONE 'self-contained' source file.
%
% ================= IF YOU HAVE QUESTIONS =======================
% Questions regarding the SIGS styles, SIGS policies and
% procedures, Conferences etc. should be sent to
% Adrienne Griscti (griscti@acm.org)
%
% Technical questions _only_ to
% Gerald Murray (murray@hq.acm.org)
% ===============================================================
%
% For tracking purposes - this is V2.0 - May 2012

\documentclass{sig-alternate}

% \usepackage[T1]{fontenc}
% \usepackage{charter}
% \usepackage[expert]{mathdesign}

%\renewcommand*\rmdefault{cmfib}
%\usepackage[T1]{fontenc}

% \usepackage[bitstream-charter]{mathdesign}
% \usepackage[T1]{fontenc}

%\usepackage{paratype}
%\usepackage[T1]{fontenc}

% \usepackage{fontspec}
% \usepackage{xltxtra}
% \setromanfont{Minion Pro}

\usepackage[urw-garamond]{mathdesign}
\usepackage[T1]{fontenc}

%\usepackage{libertine}
%\usepackage[libertine]{newtxmath}

% SIG Alternate copyright removal
\usepackage{etoolbox}
\makeatletter
\patchcmd{\maketitle}{\@copyrightspace}{}{}{}
\makeatother

\usepackage[usenames,dvipsnames]{color}


\begin{document}
%
% --- Author Metadata here ---
\CopyrightYear{2014} % Allows default copyright year (20XX) to be over-ridden - IF NEED BE.
%\crdata{0-12345-67-8/90/01}  % Allows default copyright data (0-89791-88-6/97/05) to be over-ridden - IF NEED BE.
% --- End of Author Metadata ---

\title{Benchmark Suite to Evaluate Performance of Distributed Stream
  Processing Platforms}
%
% You need the command \numberofauthors to handle the 'placement
% and alignment' of the authors beneath the title.
%
% For aesthetic reasons, we recommend 'three authors at a time'
% i.e. three 'name/affiliation blocks' be placed beneath the title.
%
% NOTE: You are NOT restricted in how many 'rows' of
% "name/affiliations" may appear. We just ask that you restrict
% the number of 'columns' to three.
%
% Because of the available 'opening page real-estate'
% we ask you to refrain from putting more than six authors
% (two rows with three columns) beneath the article title.
% More than six makes the first-page appear very cluttered indeed.
%
% Use the \alignauthor commands to handle the names
% and affiliations for an 'aesthetic maximum' of six authors.
% Add names, affiliations, addresses for
% the seventh etc. author(s) as the argument for the
% \additionalauthors command.
% These 'additional authors' will be output/set for you
% without further effort on your part as the last section in
% the body of your article BEFORE References or any Appendices.

\numberofauthors{2} %  in this sample file, there are a *total*
% of EIGHT authors. SIX appear on the 'first-page' (for formatting
% reasons) and the remaining two appear in the \additionalauthors section.
%
\author{
% You can go ahead and credit any number of authors here,
% e.g. one 'row of three' or two rows (consisting of one row of three
% and a second row of one, two or three).
%
% The command \alignauthor (no curly braces needed) should
% precede each author name, affiliation/snail-mail address and
% e-mail address. Additionally, tag each line of
% affiliation/address with \affaddr, and tag the
% e-mail address with \email.
%
% 1st. author
\alignauthor
Milinda Pathirage\\
       \affaddr{School of Informatics and Computing}\\
       \affaddr{Indiana University}\\
       \email{mpathira@indiana.edu}
% 2nd. author
\alignauthor
Beth Plale\\
       \affaddr{School of Informatics and }\\
       \affaddr{Indiana University}\\
       \email{plale@cs.indiana.edu}
}
% There's nothing stopping you putting the seventh, eighth, etc.
% author on the opening page (as the 'third row') but we ask,
% for aesthetic reasons that you place these 'additional authors'
% in the \additional authors block, viz.
%\additionalauthors{Additional authors: John Smith (The Th{\o}rv{\"a}ld Group,
%email: {\texttt{jsmith@affiliation.org}}) and Julius P.~Kumquat
%(The Kumquat Consortium, email: {\texttt{jpkumquat@consortium.net}}).}
%\date{30 July 1999}
% Just remember to make sure that the TOTAL number of authors
% is the number that will appear on the first page PLUS the
% number that will appear in the \additionalauthors section.

\maketitle
\begin{abstract}
Interest on real-time query processing and processing of continuous
streams of data has increased over last couple of years due to the need
of derving actionable information as soon as possible
 to be competitive in the fast moving world. As a result of the limitations
 in batch processing technologies from previous generation, distributed
 stream processing systems like Yahoo's \textit{S4}, Twitter's \textit{Storm}, \textit{Spark
 Streaming} were introduced into the fast growing Big Data eco-system.
 Even though there are various different stream processing platforms
 and frameworks on top of them with different capabilities and
 characteristics, in-depth comparative studies of performance, scalability and
 reliability has never been done. Users of these system often face
 difficulties when choosing a system as a solution to a task at hand.
 In this paper we use some of the popular distributed stream
 processing use cases we identified by going through mailing list
 discussions, presentations and publication to evaluate three (
 \textit{Storm, Spark Streaming, Samza}) of the
 most popular distributed processing systems used widely today.
\end{abstract}

% A category with the (minimum) three required fields
%\category{Data streams}{Please fix.}{Please fix.}
%A category including the fourth, optional field follows...
%\category{D.2.8}{Please fix.}{Metrics}[complexity measures,
%performance measures, please fix.]

%\terms{Fix this.}

%\keywords{Fix this.}

\section{Introduction}
Real-time stream processing is widely used in various scenarios
expanding across social network monitoring, fraud detection, real-time
analytics, electronic trading, sensor networks, military applications
and internet of things applications. As a result of massive growth of internet and
online social networks and commercialization of these systems, real-time
event stream processing is employed in most of these internet service
providers to be competitive with each other. This results in new stream
 processing framrworks like Apache Storm and Apache Samza. For example
Yahoo~\cite{hortonworks:2014:storm} has one of the biggest Apache Storm deployment in the
world, Twitter is using Apache Storm in their hybrid analytics
platform TSAR~\cite{twitter:2014:tsar}. In addition to that there are
several other proprietary distributed stream processing systems from
companies like Google~\cite{akidau2013millwheel} and Microsoft~\cite{murray2013naiad}.

With this push from large organizations, and contributions from some
of these organizations like Twitter~\cite{toshniwal2014storm},
Yahoo~\cite{neumeyer2010s4}, LinkedIn~\cite{asf:2014:samza} and as a
result of research~\cite{zaharia2013discretized}
from academic institutions like University of California, Berkeley
there are multiple popular open source distributed processing frameworks
available today. And lots of other organizations have started to utilize
 these event stream processing frameworks in their IT infrastructures.
 \textcolor{Red}{\textbf{Fix above.}}

Due to various differences in business requirements and existing
IT infrastructure in place, above mentioned event stream processing
systems were designed and implemented with different properties. They
each support different programming models, has different gurantees about
 scalability, fault tolerance, and etc. And some~\cite{asf:2014:samza} were implemented based on
 specific messaging infrastructures. New users, often faced various difficulties
 when trying to select one of these technologies to solve their own problems.
 This is mainly due to lack of comparative studies on these systems.

 \textcolor{Red}{\textbf{Talked about existing streaming benchmarks.}}

 \textcolor{Red}{\textbf{Define what we are going to do.}}

 \textcolor{Red}{\textbf{Our contributions.}}

 \textcolor{Red}{\textbf{Structure of the paper.}}

 \subsection{Event Stream Processing Tasks}


From~\cite{streamdrill:presentation}

 \textbf{Tasks by Complexity(Increasing Order)}

 \begin{itemize}
  \item Counting, Averages and Count Distinct
  \item Profiles and Histograms
  \item Trends
  \item Outliers and Fraud Detection
  \item Prediction (churn, failure)
 \end{itemize}

\textbf{Tasks by Latency(Decreasing Order)}
\begin{itemize}
 \item Reporting
 \item Visualization and Monitoring
 \item Optimizing, Personalizations
 \item Control
\end{itemize}


\begin{itemize}

  \item Why stream processing is widely used now?
  \item In which domains, stream processing is used.
  \item Why it is important to benchmark or do a emperical study of
    stream processing engines.
  \item What are our contributions
\end{itemize}

%
% The following two commands are all you need in the
% initial runs of your .tex file to
% produce the bibliography for the citations in your paper.
\bibliographystyle{abbrv}
\bibliography{main}  % sigproc.bib is the name of the Bibliography in this case
% You must have a proper ".bib" file
%  and remember to run:
% latex bibtex latex latex
% to resolve all references
%
% ACM needs 'a single self-contained file'!
%
\end{document}
